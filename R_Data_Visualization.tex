\documentclass[]{article}
\usepackage{lmodern}
\usepackage{amssymb,amsmath}
\usepackage{ifxetex,ifluatex}
\usepackage{fixltx2e} % provides \textsubscript
\ifnum 0\ifxetex 1\fi\ifluatex 1\fi=0 % if pdftex
  \usepackage[T1]{fontenc}
  \usepackage[utf8]{inputenc}
\else % if luatex or xelatex
  \ifxetex
    \usepackage{mathspec}
  \else
    \usepackage{fontspec}
  \fi
  \defaultfontfeatures{Ligatures=TeX,Scale=MatchLowercase}
\fi
% use upquote if available, for straight quotes in verbatim environments
\IfFileExists{upquote.sty}{\usepackage{upquote}}{}
% use microtype if available
\IfFileExists{microtype.sty}{%
\usepackage{microtype}
\UseMicrotypeSet[protrusion]{basicmath} % disable protrusion for tt fonts
}{}
\usepackage[margin=1in]{geometry}
\usepackage{hyperref}
\hypersetup{unicode=true,
            pdftitle={R - Datavisualisatie},
            pdfauthor={J.J. van Nijnatten},
            pdfborder={0 0 0},
            breaklinks=true}
\urlstyle{same}  % don't use monospace font for urls
\usepackage{color}
\usepackage{fancyvrb}
\newcommand{\VerbBar}{|}
\newcommand{\VERB}{\Verb[commandchars=\\\{\}]}
\DefineVerbatimEnvironment{Highlighting}{Verbatim}{commandchars=\\\{\}}
% Add ',fontsize=\small' for more characters per line
\usepackage{framed}
\definecolor{shadecolor}{RGB}{248,248,248}
\newenvironment{Shaded}{\begin{snugshade}}{\end{snugshade}}
\newcommand{\KeywordTok}[1]{\textcolor[rgb]{0.13,0.29,0.53}{\textbf{{#1}}}}
\newcommand{\DataTypeTok}[1]{\textcolor[rgb]{0.13,0.29,0.53}{{#1}}}
\newcommand{\DecValTok}[1]{\textcolor[rgb]{0.00,0.00,0.81}{{#1}}}
\newcommand{\BaseNTok}[1]{\textcolor[rgb]{0.00,0.00,0.81}{{#1}}}
\newcommand{\FloatTok}[1]{\textcolor[rgb]{0.00,0.00,0.81}{{#1}}}
\newcommand{\ConstantTok}[1]{\textcolor[rgb]{0.00,0.00,0.00}{{#1}}}
\newcommand{\CharTok}[1]{\textcolor[rgb]{0.31,0.60,0.02}{{#1}}}
\newcommand{\SpecialCharTok}[1]{\textcolor[rgb]{0.00,0.00,0.00}{{#1}}}
\newcommand{\StringTok}[1]{\textcolor[rgb]{0.31,0.60,0.02}{{#1}}}
\newcommand{\VerbatimStringTok}[1]{\textcolor[rgb]{0.31,0.60,0.02}{{#1}}}
\newcommand{\SpecialStringTok}[1]{\textcolor[rgb]{0.31,0.60,0.02}{{#1}}}
\newcommand{\ImportTok}[1]{{#1}}
\newcommand{\CommentTok}[1]{\textcolor[rgb]{0.56,0.35,0.01}{\textit{{#1}}}}
\newcommand{\DocumentationTok}[1]{\textcolor[rgb]{0.56,0.35,0.01}{\textbf{\textit{{#1}}}}}
\newcommand{\AnnotationTok}[1]{\textcolor[rgb]{0.56,0.35,0.01}{\textbf{\textit{{#1}}}}}
\newcommand{\CommentVarTok}[1]{\textcolor[rgb]{0.56,0.35,0.01}{\textbf{\textit{{#1}}}}}
\newcommand{\OtherTok}[1]{\textcolor[rgb]{0.56,0.35,0.01}{{#1}}}
\newcommand{\FunctionTok}[1]{\textcolor[rgb]{0.00,0.00,0.00}{{#1}}}
\newcommand{\VariableTok}[1]{\textcolor[rgb]{0.00,0.00,0.00}{{#1}}}
\newcommand{\ControlFlowTok}[1]{\textcolor[rgb]{0.13,0.29,0.53}{\textbf{{#1}}}}
\newcommand{\OperatorTok}[1]{\textcolor[rgb]{0.81,0.36,0.00}{\textbf{{#1}}}}
\newcommand{\BuiltInTok}[1]{{#1}}
\newcommand{\ExtensionTok}[1]{{#1}}
\newcommand{\PreprocessorTok}[1]{\textcolor[rgb]{0.56,0.35,0.01}{\textit{{#1}}}}
\newcommand{\AttributeTok}[1]{\textcolor[rgb]{0.77,0.63,0.00}{{#1}}}
\newcommand{\RegionMarkerTok}[1]{{#1}}
\newcommand{\InformationTok}[1]{\textcolor[rgb]{0.56,0.35,0.01}{\textbf{\textit{{#1}}}}}
\newcommand{\WarningTok}[1]{\textcolor[rgb]{0.56,0.35,0.01}{\textbf{\textit{{#1}}}}}
\newcommand{\AlertTok}[1]{\textcolor[rgb]{0.94,0.16,0.16}{{#1}}}
\newcommand{\ErrorTok}[1]{\textcolor[rgb]{0.64,0.00,0.00}{\textbf{{#1}}}}
\newcommand{\NormalTok}[1]{{#1}}
\usepackage{graphicx,grffile}
\makeatletter
\def\maxwidth{\ifdim\Gin@nat@width>\linewidth\linewidth\else\Gin@nat@width\fi}
\def\maxheight{\ifdim\Gin@nat@height>\textheight\textheight\else\Gin@nat@height\fi}
\makeatother
% Scale images if necessary, so that they will not overflow the page
% margins by default, and it is still possible to overwrite the defaults
% using explicit options in \includegraphics[width, height, ...]{}
\setkeys{Gin}{width=\maxwidth,height=\maxheight,keepaspectratio}
\IfFileExists{parskip.sty}{%
\usepackage{parskip}
}{% else
\setlength{\parindent}{0pt}
\setlength{\parskip}{6pt plus 2pt minus 1pt}
}
\setlength{\emergencystretch}{3em}  % prevent overfull lines
\providecommand{\tightlist}{%
  \setlength{\itemsep}{0pt}\setlength{\parskip}{0pt}}
\setcounter{secnumdepth}{0}
% Redefines (sub)paragraphs to behave more like sections
\ifx\paragraph\undefined\else
\let\oldparagraph\paragraph
\renewcommand{\paragraph}[1]{\oldparagraph{#1}\mbox{}}
\fi
\ifx\subparagraph\undefined\else
\let\oldsubparagraph\subparagraph
\renewcommand{\subparagraph}[1]{\oldsubparagraph{#1}\mbox{}}
\fi

%%% Use protect on footnotes to avoid problems with footnotes in titles
\let\rmarkdownfootnote\footnote%
\def\footnote{\protect\rmarkdownfootnote}

%%% Change title format to be more compact
\usepackage{titling}

% Create subtitle command for use in maketitle
\newcommand{\subtitle}[1]{
  \posttitle{
    \begin{center}\large#1\end{center}
    }
}

\setlength{\droptitle}{-2em}

  \title{R - Datavisualisatie}
    \pretitle{\vspace{\droptitle}\centering\huge}
  \posttitle{\par}
    \author{J.J. van Nijnatten}
    \preauthor{\centering\large\emph}
  \postauthor{\par}
    \date{}
    \predate{}\postdate{}
  

\begin{document}
\maketitle

{
\setcounter{tocdepth}{2}
\tableofcontents
}
\subparagraph{\texorpdfstring{contact:
\href{mailto:J.J.vanNijnatten@uva.nl}{\nolinkurl{J.J.vanNijnatten@uva.nl}}}{contact: J.J.vanNijnatten@uva.nl}}\label{contact-j.j.vannijnattenuva.nl}

\subparagraph{\texorpdfstring{broncode:
\url{https://github.com/jonasvannijnatten/R_Data_Visualization}}{broncode: https://github.com/jonasvannijnatten/R\_Data\_Visualization}}\label{broncode-httpsgithub.comjonasvannijnattenr_data_visualization}

\section{Het doel van
datavisualisatie}\label{het-doel-van-datavisualisatie}

\subsection{Wat wil je weergeven?}\label{wat-wil-je-weergeven}

\subsection{Hoe kies je de juiste manier van weergeven? (exp.
design)}\label{hoe-kies-je-de-juiste-manier-van-weergeven-exp.-design}

\section{Essentiele onderdelen van
datavisualisatie}\label{essentiele-onderdelen-van-datavisualisatie}

\subsection{gemiddelde}\label{gemiddelde}

\subsection{spreiding}\label{spreiding}

\subsection{legenda}\label{legenda}

\subsection{titel}\label{titel}

\section{Introductie GGplot2 package}\label{introductie-ggplot2-package}

\subsection{Installatie}\label{installatie}

Het package downloaden \& installeren:

\begin{Shaded}
\begin{Highlighting}[]
\KeywordTok{install.packages}\NormalTok{(}\DataTypeTok{pkgs=}\StringTok{"ggplot2"}\NormalTok{, }\DataTypeTok{repos =} \StringTok{"https://www.freestatistics.org/cran/"}\NormalTok{)}
\KeywordTok{install.packages}\NormalTok{(}\DataTypeTok{pkgs=}\StringTok{"Hmisc"}\NormalTok{,   }\DataTypeTok{repos =} \StringTok{"https://www.freestatistics.org/cran/"}\NormalTok{)}
\end{Highlighting}
\end{Shaded}

Het package library activeren:

\begin{Shaded}
\begin{Highlighting}[]
\KeywordTok{library}\NormalTok{(}\DataTypeTok{package=}\StringTok{"Hmisc"}\NormalTok{)}
\KeywordTok{library}\NormalTok{(}\DataTypeTok{package=}\StringTok{"ggplot2"}\NormalTok{)}
\end{Highlighting}
\end{Shaded}

\begin{center}\rule{0.5\linewidth}{\linethickness}\end{center}

\newpage

\subsection{Opbouw van figuren}\label{opbouw-van-figuren}

\section{Voorbeelden}\label{voorbeelden}

\subsection{Data-inspectie}\label{data-inspectie}

\begin{center}\rule{0.5\linewidth}{\linethickness}\end{center}

\newpage

\subsection{Normaliteit}\label{normaliteit}

\begin{center}\rule{0.5\linewidth}{\linethickness}\end{center}

\newpage

\subsection{T-test}\label{t-test}

\begin{Shaded}
\begin{Highlighting}[]
\CommentTok{# generate data}
\NormalTok{group1 =}\StringTok{ }\KeywordTok{rnorm}\NormalTok{(}\DataTypeTok{n =} \DecValTok{40}\NormalTok{, }\DataTypeTok{mean =} \DecValTok{25}\NormalTok{, }\DataTypeTok{sd =} \FloatTok{6.5}\NormalTok{)}
\NormalTok{group2 =}\StringTok{ }\KeywordTok{rnorm}\NormalTok{(}\DataTypeTok{n =} \DecValTok{40}\NormalTok{, }\DataTypeTok{mean =} \DecValTok{35}\NormalTok{, }\DataTypeTok{sd =} \FloatTok{6.5}\NormalTok{)}
\NormalTok{data.wide =}\StringTok{ }\KeywordTok{data.frame}\NormalTok{(group1, group2)}
\end{Highlighting}
\end{Shaded}

\begin{Shaded}
\begin{Highlighting}[]
\CommentTok{# reshape data}
\NormalTok{data.long =}\StringTok{ }\KeywordTok{reshape}\NormalTok{(}\DataTypeTok{data =} \NormalTok{data.wide, }
                    \DataTypeTok{direction =} \StringTok{"long"}
                    \NormalTok{, }\DataTypeTok{varying =} \KeywordTok{c}\NormalTok{(}\StringTok{"group1"}\NormalTok{, }\StringTok{"group2"}\NormalTok{)}
                    \NormalTok{, }\DataTypeTok{v.names =} \StringTok{"score"}
                    \NormalTok{, }\DataTypeTok{times   =} \KeywordTok{c}\NormalTok{(}\StringTok{'pre'}\NormalTok{, }\StringTok{'post'}\NormalTok{) }
                      \NormalTok{)}
\end{Highlighting}
\end{Shaded}

\begin{Shaded}
\begin{Highlighting}[]
\CommentTok{# plot means and standard deviations}
\KeywordTok{ggplot}\NormalTok{(data.long, }\KeywordTok{aes}\NormalTok{(}\DataTypeTok{x=}\NormalTok{time, }\DataTypeTok{y=}\NormalTok{score, }\DataTypeTok{fill=}\NormalTok{time) ) +}\StringTok{ }
\StringTok{  }\KeywordTok{geom_bar}     \NormalTok{( }\DataTypeTok{stat =} \StringTok{"summary"}\NormalTok{, }\DataTypeTok{fun.y    =} \StringTok{"mean"} \NormalTok{) +}\StringTok{ }
\StringTok{  }\KeywordTok{geom_errorbar}\NormalTok{( }\DataTypeTok{stat =} \StringTok{"summary"}\NormalTok{, }\DataTypeTok{fun.data =} \StringTok{"mean_sdl"}\NormalTok{, }\DataTypeTok{fun.args =} \DecValTok{1}\NormalTok{, }\DataTypeTok{width =} \FloatTok{0.3} \NormalTok{) +}\StringTok{ }
\StringTok{  }\KeywordTok{geom_point}   \NormalTok{( }\DataTypeTok{position=}\KeywordTok{position_jitter}\NormalTok{(}\DataTypeTok{width=}\NormalTok{.}\DecValTok{1}\NormalTok{) )}
\end{Highlighting}
\end{Shaded}

\includegraphics{R_Data_Visualization_files/figure-latex/unnamed-chunk-5-1.pdf}
To plot standard errors instead of standard deviations replace
``mean\_sdl'' with ``mean\_se'', and it is common use to plot 2 (or
1.96) times the standard error to get an 95\% confidence interval, so
replace ``fun.arg = 1'' with ``fun.arg = 2''.

\begin{Shaded}
\begin{Highlighting}[]
\KeywordTok{ggplot}\NormalTok{(data.long, }\KeywordTok{aes}\NormalTok{(}\DataTypeTok{x=}\NormalTok{time, }\DataTypeTok{y=}\NormalTok{score, }\DataTypeTok{fill=}\NormalTok{time) ) +}\StringTok{ }
\StringTok{  }\KeywordTok{geom_bar}     \NormalTok{( }\DataTypeTok{stat =} \StringTok{"summary"}\NormalTok{, }\DataTypeTok{fun.y    =} \StringTok{"mean"} \NormalTok{) +}\StringTok{ }
\StringTok{  }\KeywordTok{geom_errorbar}\NormalTok{( }\DataTypeTok{stat =} \StringTok{"summary"}\NormalTok{, }\DataTypeTok{fun.data =} \StringTok{"mean_se"}\NormalTok{, }\DataTypeTok{fun.args =} \DecValTok{2}\NormalTok{, }\DataTypeTok{width =} \FloatTok{0.3} \NormalTok{) +}\StringTok{ }
\StringTok{  }\KeywordTok{geom_jitter}  \NormalTok{( }\DataTypeTok{width =} \NormalTok{.}\DecValTok{05} \NormalTok{)}
\end{Highlighting}
\end{Shaded}

\includegraphics{R_Data_Visualization_files/figure-latex/unnamed-chunk-6-1.pdf}

\begin{center}\rule{0.5\linewidth}{\linethickness}\end{center}

\newpage

\subsection{Correlatie}\label{correlatie}

\begin{center}\rule{0.5\linewidth}{\linethickness}\end{center}

\newpage

\subsection{Regressie}\label{regressie}

\begin{center}\rule{0.5\linewidth}{\linethickness}\end{center}

\newpage

\subsection{One-way independent samples
ANOVA}\label{one-way-independent-samples-anova}

\begin{Shaded}
\begin{Highlighting}[]
\KeywordTok{set.seed}\NormalTok{(}\DecValTok{05}\NormalTok{)   }\CommentTok{# set seed}
\NormalTok{nrofconds =}\StringTok{ }\DecValTok{3}  \CommentTok{# set number of conditions}
\NormalTok{nrofsubs  =}\StringTok{ }\DecValTok{20} \CommentTok{# set number of subjects}
\NormalTok{subj =}\StringTok{ }\KeywordTok{as.factor}\NormalTok{(}\DecValTok{1}\NormalTok{:(nrofsubs*nrofconds))      }\CommentTok{# create array with subject IDs}
\NormalTok{cond =}\StringTok{ }\KeywordTok{as.factor}\NormalTok{(}\KeywordTok{rep}\NormalTok{(LETTERS[}\DecValTok{1}\NormalTok{:nrofconds],}\DataTypeTok{each=}\NormalTok{nrofsubs))   }\CommentTok{# create array with condition values}
\NormalTok{score =}\StringTok{ }\KeywordTok{as.vector}\NormalTok{( }\KeywordTok{replicate}\NormalTok{(}
          \NormalTok{nrofconds , }\KeywordTok{rnorm}\NormalTok{(}\DataTypeTok{n =} \NormalTok{nrofsubs, }\DataTypeTok{mean =} \KeywordTok{sample}\NormalTok{(}\DecValTok{8}\NormalTok{,}\DecValTok{1}\NormalTok{)+}\DecValTok{10} \NormalTok{, }\DataTypeTok{sd =} \KeywordTok{sample}\NormalTok{(}\DecValTok{5}\NormalTok{,}\DecValTok{1}\NormalTok{) ) }
        \NormalTok{) )                                     }\CommentTok{# create array with measurement values}
\NormalTok{data.long =}\StringTok{ }\KeywordTok{data.frame}\NormalTok{(subj, cond, score);      }\CommentTok{# combine arrays into a data.frame}
\KeywordTok{rm}\NormalTok{(}\DataTypeTok{list=}\KeywordTok{setdiff}\NormalTok{(}\KeywordTok{ls}\NormalTok{(), }\KeywordTok{c}\NormalTok{(}\StringTok{"data.long"}\NormalTok{, }\StringTok{"nrofsubs"}\NormalTok{,}\StringTok{"nrofconds"}\NormalTok{))) }\CommentTok{# delete arrays}
\end{Highlighting}
\end{Shaded}

\begin{Shaded}
\begin{Highlighting}[]
\KeywordTok{ggplot}\NormalTok{(data.long, }\KeywordTok{aes}\NormalTok{(}\DataTypeTok{x=}\NormalTok{cond, }\DataTypeTok{y=}\NormalTok{score, }\DataTypeTok{fill=}\NormalTok{cond) ) +}\StringTok{ }
\StringTok{  }\KeywordTok{geom_bar}     \NormalTok{( }\DataTypeTok{stat =} \StringTok{"summary"}\NormalTok{, }\DataTypeTok{fun.y    =} \StringTok{"mean"} \NormalTok{) +}\StringTok{ }
\StringTok{  }\KeywordTok{geom_errorbar}\NormalTok{( }\DataTypeTok{stat =} \StringTok{"summary"}\NormalTok{, }\DataTypeTok{fun.data =} \StringTok{"mean_se"}\NormalTok{, }\DataTypeTok{fun.args =} \DecValTok{2}\NormalTok{, }\DataTypeTok{width =} \FloatTok{0.3} \NormalTok{) +}\StringTok{ }
\StringTok{  }\KeywordTok{geom_point}   \NormalTok{( }\DataTypeTok{position=}\KeywordTok{position_jitter}\NormalTok{(}\DataTypeTok{width =} \NormalTok{.}\DecValTok{1}\NormalTok{) )}
\end{Highlighting}
\end{Shaded}

\includegraphics{R_Data_Visualization_files/figure-latex/unnamed-chunk-8-1.pdf}

\begin{center}\rule{0.5\linewidth}{\linethickness}\end{center}

\newpage

\subsection{Factorial independent samples
ANOVA}\label{factorial-independent-samples-anova}

\begin{Shaded}
\begin{Highlighting}[]
\KeywordTok{set.seed}\NormalTok{(}\DecValTok{01}\NormalTok{)   }\CommentTok{# set seed}
\NormalTok{nrofcondsf1 =}\StringTok{ }\DecValTok{3}  \CommentTok{# set number of conditions for factor 1}
\NormalTok{nrofcondsf2 =}\StringTok{ }\DecValTok{2}  \CommentTok{# set number of conditions for factor 2}
\NormalTok{nrofsubs    =}\StringTok{ }\NormalTok{nrofcondsf1*nrofcondsf2*}\DecValTok{10} \CommentTok{# set number of subjects per condition}
\NormalTok{subj =}\StringTok{ }\KeywordTok{as.factor}\NormalTok{(}\DecValTok{1}\NormalTok{:(nrofsubs))      }\CommentTok{# create array with subject IDs}
\NormalTok{treatment =}\StringTok{ }\KeywordTok{as.factor}\NormalTok{(}\KeywordTok{rep}\NormalTok{(LETTERS[}\DecValTok{1}\NormalTok{:nrofcondsf1],}\DataTypeTok{each=}\NormalTok{nrofsubs/nrofcondsf1))   }\CommentTok{# create array witht treatment conditions}
\NormalTok{control   =}\StringTok{ }\KeywordTok{as.factor}\NormalTok{(}\KeywordTok{rep}\NormalTok{(}\KeywordTok{c}\NormalTok{(}\StringTok{"control"}\NormalTok{,}\StringTok{"experimental"}\NormalTok{),}\DataTypeTok{times=}\NormalTok{nrofsubs/nrofcondsf2))   }\CommentTok{# create array with control / experimental}
\NormalTok{score =}\StringTok{ }\KeywordTok{as.vector}\NormalTok{( }\KeywordTok{replicate}\NormalTok{(nrofcondsf1, }\KeywordTok{replicate} \NormalTok{( }
          \NormalTok{nrofcondsf2 , }\KeywordTok{rnorm}\NormalTok{(}\DataTypeTok{n =} \NormalTok{(nrofsubs/(nrofcondsf1*nrofcondsf2)), }\DataTypeTok{mean =} \KeywordTok{sample}\NormalTok{(}\DecValTok{14}\NormalTok{,}\DecValTok{1}\NormalTok{)+}\DecValTok{10} \NormalTok{, }\DataTypeTok{sd =} \KeywordTok{sample}\NormalTok{(}\DecValTok{5}\NormalTok{,}\DecValTok{1}\NormalTok{) ) }
        \NormalTok{) ) )                                    }\CommentTok{# create array with measurement values}
\NormalTok{data.long =}\StringTok{ }\KeywordTok{data.frame}\NormalTok{(subj, score, treatment, control);      }\CommentTok{# combine arrays into a data.frame}
\KeywordTok{rm}\NormalTok{(}\DataTypeTok{list=}\KeywordTok{setdiff}\NormalTok{(}\KeywordTok{ls}\NormalTok{(), }\KeywordTok{c}\NormalTok{(}\StringTok{"data.long"}\NormalTok{, }\StringTok{"nrofsubs"}\NormalTok{,}\StringTok{"nrofconds"}\NormalTok{))) }\CommentTok{# delete arrays}
\end{Highlighting}
\end{Shaded}

\begin{Shaded}
\begin{Highlighting}[]
\KeywordTok{ggplot}\NormalTok{(data.long, }\KeywordTok{aes}\NormalTok{(}\DataTypeTok{x=}\NormalTok{treatment, }\DataTypeTok{y=}\NormalTok{score, }\DataTypeTok{fill=}\NormalTok{control) ) +}\StringTok{ }
\StringTok{  }\KeywordTok{geom_bar}     \NormalTok{( }\DataTypeTok{stat =} \StringTok{"summary"}\NormalTok{, }\DataTypeTok{fun.y    =} \StringTok{"mean"} \NormalTok{, }\DataTypeTok{position =} \StringTok{"dodge"}\NormalTok{) +}\StringTok{ }
\StringTok{  }\KeywordTok{geom_errorbar}\NormalTok{( }\DataTypeTok{stat =} \StringTok{"summary"}\NormalTok{, }\DataTypeTok{fun.data =} \StringTok{"mean_se"}\NormalTok{, }\DataTypeTok{fun.args =} \DecValTok{2}\NormalTok{, }\DataTypeTok{width =} \FloatTok{0.3}\NormalTok{, }
                 \DataTypeTok{position =} \KeywordTok{position_dodge}\NormalTok{(}\DataTypeTok{width=}\NormalTok{.}\DecValTok{9}\NormalTok{) ) +}\StringTok{ }
\StringTok{  }\KeywordTok{geom_point}  \NormalTok{( }\DataTypeTok{position =} \KeywordTok{position_jitterdodge}\NormalTok{(}\DataTypeTok{jitter.width =} \NormalTok{.}\DecValTok{1}\NormalTok{) )}
\end{Highlighting}
\end{Shaded}

\includegraphics{R_Data_Visualization_files/figure-latex/unnamed-chunk-10-1.pdf}

\begin{center}\rule{0.5\linewidth}{\linethickness}\end{center}

\newpage  

\subsection{One-way repeated measures
ANOVA}\label{one-way-repeated-measures-anova}

Generate dataset

\begin{Shaded}
\begin{Highlighting}[]
\KeywordTok{set.seed}\NormalTok{(}\DecValTok{01}\NormalTok{)   }\CommentTok{# set seed}
\NormalTok{nrofsubs  =}\StringTok{ }\DecValTok{20} \CommentTok{# set number of subjects}
\NormalTok{nrofconds =}\StringTok{ }\DecValTok{3}  \CommentTok{# set number of conditions}
\NormalTok{subj =}\StringTok{ }\KeywordTok{as.factor}\NormalTok{(}\KeywordTok{rep}\NormalTok{(}\DecValTok{1}\NormalTok{:nrofsubs,nrofconds))      }\CommentTok{# create array with subject IDs}
\NormalTok{cond =}\StringTok{ }\KeywordTok{as.factor}\NormalTok{(}\KeywordTok{rep}\NormalTok{(LETTERS[}\DecValTok{1}\NormalTok{:nrofconds],}\DataTypeTok{each=}\NormalTok{nrofsubs))   }\CommentTok{# create array with condition values}
\NormalTok{score =}\StringTok{ }\KeywordTok{as.vector}\NormalTok{( }\KeywordTok{replicate}\NormalTok{(}
          \NormalTok{nrofconds , }\KeywordTok{rnorm}\NormalTok{(}\DataTypeTok{n =} \NormalTok{nrofsubs, }\DataTypeTok{mean =} \KeywordTok{sample}\NormalTok{(}\DecValTok{8}\NormalTok{,}\DecValTok{1}\NormalTok{)+}\DecValTok{10} \NormalTok{, }\DataTypeTok{sd =} \KeywordTok{sample}\NormalTok{(}\DecValTok{5}\NormalTok{,}\DecValTok{1}\NormalTok{) ) }
        \NormalTok{) )                                     }\CommentTok{# create array with measurement values}
\NormalTok{data.long =}\StringTok{ }\KeywordTok{data.frame}\NormalTok{(subj, cond, score);      }\CommentTok{# combine arrays into a data.frame}
\KeywordTok{rm}\NormalTok{(}\DataTypeTok{list=}\KeywordTok{setdiff}\NormalTok{(}\KeywordTok{ls}\NormalTok{(), }\KeywordTok{c}\NormalTok{(}\StringTok{"data.long"}\NormalTok{, }\StringTok{"nrofsubs"}\NormalTok{,}\StringTok{"nrofconds"}\NormalTok{))) }\CommentTok{# delete arrays}
\end{Highlighting}
\end{Shaded}

\begin{Shaded}
\begin{Highlighting}[]
\KeywordTok{ggplot}\NormalTok{(data.long, }\KeywordTok{aes}\NormalTok{(}\DataTypeTok{x=}\NormalTok{cond, }\DataTypeTok{y=}\NormalTok{score, }\DataTypeTok{group=}\DecValTok{1}\NormalTok{, }\DataTypeTok{colour=}\NormalTok{subj)) +}\StringTok{ }
\StringTok{  }\KeywordTok{geom_point}   \NormalTok{() +}
\StringTok{  }\KeywordTok{geom_line}    \NormalTok{( }\DataTypeTok{linetype=} \StringTok{"dashed"}\NormalTok{, }\KeywordTok{aes}\NormalTok{(}\DataTypeTok{group=}\NormalTok{subj) ) +}
\StringTok{  }\KeywordTok{geom_line}    \NormalTok{( }\DataTypeTok{stat =} \StringTok{"summary"}\NormalTok{, }\DataTypeTok{fun.y    =} \StringTok{"mean"}\NormalTok{,    }\DataTypeTok{size=}\DecValTok{2}\NormalTok{, }\DataTypeTok{colour =} \StringTok{"black"}\NormalTok{, }\DataTypeTok{linetype=} \StringTok{"solid"}\NormalTok{) +}
\StringTok{  }\KeywordTok{geom_point}   \NormalTok{( }\DataTypeTok{stat =} \StringTok{"summary"}\NormalTok{, }\DataTypeTok{fun.y    =} \StringTok{"mean"}\NormalTok{,    }\DataTypeTok{size=}\DecValTok{2}\NormalTok{, }\DataTypeTok{colour =} \StringTok{"black"} \NormalTok{) +}
\StringTok{  }\KeywordTok{geom_errorbar}\NormalTok{( }\DataTypeTok{stat =} \StringTok{"summary"}\NormalTok{, }\DataTypeTok{fun.data =} \StringTok{"mean_se"}\NormalTok{, }\DataTypeTok{size=}\DecValTok{1} \NormalTok{, }\DataTypeTok{fun.args =} \DecValTok{2}\NormalTok{, }\DataTypeTok{width =} \FloatTok{0.3}\NormalTok{)}
\end{Highlighting}
\end{Shaded}

\includegraphics{R_Data_Visualization_files/figure-latex/unnamed-chunk-12-1.pdf}

\begin{center}\rule{0.5\linewidth}{\linethickness}\end{center}

\newpage

\subsection{Factorial repeated measures
ANOVA}\label{factorial-repeated-measures-anova}

\begin{Shaded}
\begin{Highlighting}[]
\KeywordTok{set.seed}\NormalTok{(}\DecValTok{02}\NormalTok{)  }\CommentTok{# set seed}
\NormalTok{nrofcondsf1 =}\StringTok{ }\DecValTok{3}  \CommentTok{# set number of conditions for factor 1}
\NormalTok{nrofcondsf2 =}\StringTok{ }\DecValTok{2}  \CommentTok{# set number of conditions for factor 2}
\NormalTok{nrofsubs    =}\StringTok{ }\DecValTok{10} \CommentTok{# set number of subjects }
\NormalTok{subj =}\StringTok{ }\KeywordTok{as.factor}\NormalTok{(}\KeywordTok{rep}\NormalTok{(}\DecValTok{1}\NormalTok{:(nrofsubs),}\DataTypeTok{times=}\NormalTok{nrofcondsf1*nrofcondsf2))    }\CommentTok{# create array with subject IDs}
\NormalTok{treatment =}\StringTok{ }\KeywordTok{as.factor}\NormalTok{(}\KeywordTok{rep}\NormalTok{(LETTERS[}\DecValTok{1}\NormalTok{:nrofcondsf1],}\DataTypeTok{each=}\NormalTok{nrofsubs*nrofcondsf2))   }\CommentTok{# create array witht treatment conditions}
\NormalTok{control   =}\StringTok{ }\KeywordTok{as.factor}\NormalTok{(}\KeywordTok{rep}\NormalTok{(}\KeywordTok{rep}\NormalTok{(}\KeywordTok{c}\NormalTok{(}\StringTok{"control"}\NormalTok{,}\StringTok{"experimental"}\NormalTok{),}\DataTypeTok{each=}\NormalTok{nrofsubs),}\DataTypeTok{times=}\NormalTok{nrofcondsf1))   }\CommentTok{# create array with control / experimental}
\NormalTok{score =}\StringTok{ }\KeywordTok{as.vector}\NormalTok{( }\KeywordTok{replicate}\NormalTok{(nrofcondsf1, }\KeywordTok{replicate}\NormalTok{(nrofcondsf2,    }\CommentTok{# create array with measurement values}
                             \KeywordTok{rnorm}\NormalTok{(}\DataTypeTok{n =} \NormalTok{(nrofsubs), }\DataTypeTok{mean =} \KeywordTok{sample}\NormalTok{(}\DecValTok{14}\NormalTok{,}\DecValTok{1}\NormalTok{)+}\DecValTok{10} \NormalTok{, }\DataTypeTok{sd =} \KeywordTok{sample}\NormalTok{(}\DecValTok{5}\NormalTok{,}\DecValTok{1}\NormalTok{)                }
                                   \NormalTok{) ) ) )                  }
\NormalTok{data.long =}\StringTok{ }\KeywordTok{data.frame}\NormalTok{(subj, score, treatment, control);      }\CommentTok{# combine arrays into a data.frame}
\KeywordTok{rm}\NormalTok{(}\DataTypeTok{list=}\KeywordTok{setdiff}\NormalTok{(}\KeywordTok{ls}\NormalTok{(), }\KeywordTok{c}\NormalTok{(}\StringTok{"data.long"}\NormalTok{, }\StringTok{"nrofsubs"}\NormalTok{,}\StringTok{"nrofconds"}\NormalTok{))) }\CommentTok{# delete arrays}
\end{Highlighting}
\end{Shaded}

\begin{Shaded}
\begin{Highlighting}[]
\KeywordTok{ggplot}\NormalTok{(data.long, }\KeywordTok{aes}\NormalTok{(}\DataTypeTok{x=}\NormalTok{treatment, }\DataTypeTok{y=}\NormalTok{score, }\DataTypeTok{group=}\NormalTok{control, }\DataTypeTok{colour=}\NormalTok{control)) +}\StringTok{ }
\StringTok{  }\KeywordTok{geom_point}   \NormalTok{(}\DataTypeTok{size=}\DecValTok{1}\NormalTok{)  +}
\StringTok{  }\KeywordTok{geom_line}    \NormalTok{(}\DataTypeTok{linetype=}\StringTok{"dashed"} \NormalTok{,}\KeywordTok{aes}\NormalTok{(}\DataTypeTok{group=}\KeywordTok{interaction}\NormalTok{(subj,control), }\DataTypeTok{alpha=}\NormalTok{.}\DecValTok{5}\NormalTok{) ) +}
\StringTok{  }\KeywordTok{geom_line}    \NormalTok{( }\DataTypeTok{stat =} \StringTok{"summary"}\NormalTok{, }\DataTypeTok{fun.y    =} \StringTok{"mean"}   \NormalTok{, }\DataTypeTok{size=}\FloatTok{1.5} \NormalTok{)  +}
\StringTok{  }\KeywordTok{geom_point}   \NormalTok{( }\DataTypeTok{stat =} \StringTok{"summary"}\NormalTok{, }\DataTypeTok{fun.y    =} \StringTok{"mean"}   \NormalTok{, }\DataTypeTok{size=}\DecValTok{2}   \NormalTok{) +}
\StringTok{  }\KeywordTok{geom_errorbar}\NormalTok{( }\DataTypeTok{stat =} \StringTok{"summary"}\NormalTok{, }\DataTypeTok{fun.data =} \StringTok{"mean_se"}\NormalTok{, }\DataTypeTok{size=}\DecValTok{1} \NormalTok{, }\DataTypeTok{fun.args =} \DecValTok{2}\NormalTok{, }\DataTypeTok{width =} \FloatTok{0.3}\NormalTok{)}
\end{Highlighting}
\end{Shaded}

\includegraphics{R_Data_Visualization_files/figure-latex/unnamed-chunk-14-1.pdf}

\begin{center}\rule{0.5\linewidth}{\linethickness}\end{center}


\end{document}
